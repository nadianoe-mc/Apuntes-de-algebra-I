Relaciones antisimétricas.

la definición dice que una relación es antisimétrica si y solo si $aRb y bRa \Rightarrow a=b$.

Analicemos con detenimiento la definición.
La premisa indica que si es verdad que en una relación esté un par ordenado y su elmento simétrico, entonces se debe tratar de un mismo elemento relacionado con sí mismo. 

Según las reglas lógicas, la implicación anterior es equivalente a decir
$si es a distinto de b, entonces negamos la conjunción (aRb y bRa)$
negar la conjunción es equivalente a decir [(a no se relaciona con b) o (b no se relaciona con a)]. Entonces, s en la implicación deducida puede ser de tres formas distintas, 

1) que a no se relaciona con b
2) que b no se relaciona con a
y
3) 1) y 2) simultaneamente.



