clase 3, práctica

\section{ejercicio 1}

Sea A un conjunto. A es igual a \{1, 2, 3\}. Sea R una relación en P(A), definida como $ BRC \Leftrightarrow B \subseteq C$.\\


Pruebe que R es una relación de orden.\\

Para probar que R es una relación de orden, debemos probar que ésta es una relación reflexiva, antisimétrica  transitiva.\\

\subsection{Veamos si la relación R es reflexiva:}

Recordá que para que sea reflexiva, debe suceder que cada uno de los elementos pertenecientes a P(A) estén relacionados consigo mismos; y, en este caso particular, deben estar incluídos en sí mismos. (Ya que esa es la condición que cumplen tooodos los pares ordenados pertenecientes a la relación R.\\

Entonces, debemos ver si H $\subseteq$ H, siendo H un elemento cualquiera de P(A).

Efectivamente, sucede? cómo lo podemos probar?

pensemos, si tomamos a F como un elemento cualquiera de P(A), éste F está incluído en sí mismo? es decir, $F \subseteq F$?? Sí! Ya que todo conjunto está incluído en sí mismo. Veamos que por definición de inclusión de conjuntos, se cumple que F esté incluído en sí mismo ya que todos elementos de F están en F, por lo tanto $F \subseteq f$. Entonces, R es una relación reflexiva. 

\subsection{Veamos si la relación R es antisimétrica:}

Para ver esto tenemos que ver que si $DRE y ERD \Rightarrow D = E$

Esto sucede? Veamos. Si tomamos $J$ y $T$, dos elementos de P(A), tales que $JRT$ y $TRJ$; entonces, por definición de la relación $J \subseteq T$ y $ T\subseteq J$, entonces por definición de igualdad de subconjuntos, deducimos que $T=J$. Por lo tanto, R también es antisimétrica.




